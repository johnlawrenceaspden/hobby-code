\documentclass[14pt,a4paper]{memoir}
\usepackage[margin=0.1in]{geometry}

\begin{document}

\chapter*{Singing Lessons with James Eisner}

\section*{First Lesson: Breathing}

\begin{itemize}
\item Balance evenly on both feet, and between forward and back

\item Back of neck straight not tilted back (crushes larynx)

\item Drop Jaw (which I think was another way of saying don't tilt head back rather than an instruction to do anything with jaw itself)

\item Shield and Ring / Thyroid and Cricoid cartilages

\item Singing is the opposite of swallowing. When you swallow, everything closes up, you want it all as open as possible.
So can practice swallowing and learn what it feels like to relax everything.

\item Drinking water is good to keep everything lubricated

\item Cat and Cow exercise, on all fours arch and hollow back: inhale to midpoint and exhale to both extremes.
makes you conscious of how your breathing apparatus is working and the feeling of things filling up with air

\item Back Breathe: hang off door handle, tuck in tailbone, let go of lower back, feel expansion under ribs at back
expansion goes into pelvis front and back and side. 

\item Reflexive Breathe: sink down and cross arms while exhaling, stand up and raise arms to Y-shape, allow breathing in to happen automatically. (can extend the exhale phase to hand-walking to flat)
  
\item While making a note, tap all inflating areas (breastbone, ribs, abdomen, sides, lower back) and feel effect in voicebox (if connected, note should rise)
  
\item Lower spine is over-concave, stand straight, get taller.
Alexander technique: imagine you're suspended from the ceiling by a string to the top of your head
If you get this right then it improves the connection to the various abdomen areas as tested above.

\item Sing on 'ng' to put all air through nasopharynx

\item Eurythmy (actions as physical mnemonics for various notes, can't find these on internet?!)

\item Use the French time names, but h is better than s for rests, should be able to find them plus recordings on internet. (I like ta-ka-di-mi better!)

\item Metrical Hierarchy (stress beat 1, and also 3 when counting in fours)

\item Singing through a straw is resistance training for voice
  
\end{itemize}

\section*{Second Lesson: Falsetto}

\begin{itemize}

\item We did a long physical warm up which I don't have notes for

warm up (stretch vocal folds, hydrate them)

\item James thinks I'm a (high?) baritone ; I have a comfy middle range top and bottom not so comfy


\item low notes/''chest voice'', vocalis muscle in airstream is a relaxation oscillator (whoopee cushion!)

\item falsetto/extreme 'head voice'

crico aretenoid cartileges tighten, tenses lip of vocal fold (vocal ligament)

crico thyroid muscle

work together to tension vocal fold

\item wish to blend the two voices head and chest, so work on upper part

  
\item  Saying/singing \textbf{'....ing ing ing ing....'} ; causes transition at \textbf{back of tongue, taps against soft palate}

this helps with high notes


\item \textbf {fragrant breath}

inhale nice smoke (clan, balkan sobranie)

\textbf {soft palate lifts}

\textbf {back of tongue lifts}

they are connected to the back of the thyroid cartilege

aiming for a non-breathy falsetto

\item I am covering the sound:

\textbf{externalise sound} / will it out there

\item use the light end of the voice,  thin folds, then attain G4 easily

\item sing on a smile / lift facial muscles

\item relax jaw


\end{itemize}


\section*{Third Lesson:Sainte Nicholaes}

I was worrying that the high note in 'Sainte Nicholaes' is too high (it's D4, but it feels like a stretch)
And also worrying that I sometimes run out of breath in second phrase 'schone (gasp) hus'
This doesn't happen if I sing the song a couple of tones down, then I have effortless control, so is it a technical issue?


\begin{itemize}
\item Running out of breath
  Imagine a bellows, there are three ways it can not have enough air
  \begin{itemize}
  \item how much air have you got to start with
  \item flow rate
  \item stop too early
  \end{itemize}

  \item online 'singing zone dot com' resource has lots of stuff, is pay, James can lend me videos

  \item Hum Sainte Nicholaes on 'ng', to put tone into nose, isolate vocal cords, focus on tone stream
    Singing 'ng' makes tongue close air way, so can use that to get sounds without breath

  \item Can also do this sliding between notes, sirening

  \item Vowel stuff is separate, happens in mouth.

  \item Try 'Mirening', sing all vowels and consonants in mouth, but not allowing air to pass through mouth, so air all going through nasopharynx
  while tongue lips teeth all perform their actions


  \item Muscular dissociation is getting mouth parts to act independently of what larynx / cords are doing, like patting head and rubbing stomach

  \item Since song is easy two tones down, start off two tones down and go up a semitone keeping feelings until get to correct pitch

  \item Emphasize consonants


  \item Cross country skiing warmup exercise is to help with externalizing sounds
    keep whooping hey higher and higher with swinging arms, also imagining back-against-door exercise

  \item Back against door exercise is about posture, also allows one to think about sound 'flowing through one'

  \item Difference between 'covered' and 'uncovered' sound (could also say 'externalized', or 'brash')
    is about the high resonances
    Resonating chamber is between tongue and soft palate
    if that's large, then resonant freq is lower, and sound is purer (covered, monk-like)
    if it's small, then there are higher frequencies, and the sound is 'externalized', 'brash'
    Estil singing calls the covered sound 'sob', (sounds like someone sobbing?)

  \item Try to sing 'ng' on a yawn, yawn lowers larynx (can feel this by putting hand against throat), ng raises back of tongue (very hard, practise)

  \item Deep gargling washes vocal cords (this cleared my feeling of phlegm stuck in throat, really helped)

  \item singing through a straw is great

  \item ``Kiss lips'' help to project sound forward

  \item I am over-neutral with vowels and consonants, (too close to just being a tone stream?)
  To get them working practice:

  ``The tip of the tongue and the teeth and the lips''
  
  ``Pop a cat a petal Copper Plated Kettle''

  \item Practice emphasising articulation

    Articulation is anything involving movement of mouth parts

  \item James makes the same criticism even when I'm singing low:

    I'm just not bothering to pronounce the words, just making notes, flaccid, not making words.

  \item MUSCLE DISSOCIATION IS THE THING TODAY

    Mouth and tongue and lips should be independent of what larynx etc are doing

  \item So practise all mouth movements, emphasise them, and keep those feelings while singing

  \item E.g. just practise saying the song, and try to keep saying it while singing.

  \item Stand against door and imagine a string (connecting to top of head?) 'tea bag exercise'

    Say song, emphasising all consonants and vowels

    Hum song on 'ng'

    Combine the two, keep thinking about mouth movements, let larynx do its own thing

  \item Smile enough to uncover your top teeth, imagine that the points of your canines are 'tweeters' (small speakers)


  \item I then asked if there was a way to make the low C (C2) I can sometimes get be more certain.

  James doesn't know much about the low end, but does know one helpful exercise:

  Start off on all fours, while singing low go back into 'Child Position', if voice becomes breathy, go up a bit and regain control.
  

\end{itemize}





\end{document}

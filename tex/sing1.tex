\documentclass[14pt,a4paper]{memoir}
\usepackage[margin=0.1in]{geometry}

\begin{document}

\chapter*{Singing Lesson with James Eisner}

First Lesson

\begin{itemize}
\item Balance evenly on both feet, and between forward and back

\item Back of neck straight not tilted back (crushes larynx)

\item Drop Jaw (which I think was another way of saying don't tilt head back rather than an instruction to do anything with jaw itself)

\item Shield and Ring / Thyroid and Cricoid cartilages

\item Singing is the opposite of swallowing. When you swallow, everything closes up, you want it all as open as possible.
So can practice swallowing and learn what it feels like to relax everything.

\item Drinking water is good to keep everything lubricated

\item Cat and Cow exercise, on all fours arch and hollow back: inhale to midpoint and exhale to both extremes.
makes you conscious of how your breathing apparatus is working and the feeling of things filling up with air

\item Back Breathe: hang off door handle, tuck in tailbone, let go of lower back, feel expansion under ribs at back
expansion goes into pelvis front and back and side. 

\item Reflexive Breathe: sink down and cross arms while exhaling, stand up and raise arms to Y-shape, allow breathing in to happen automatically. (can extend the exhale phase to hand-walking to flat)
  
\item While making a note, tap all inflating areas (breastbone, ribs, abdomen, sides, lower back) and feel effect in voicebox (if connected, note should rise)
  
\item Lower spine is over-concave, stand straight, get taller.
Alexander technique: imagine you're suspended from the ceiling by a string to the top of your head
If you get this right then it improves the connection to the various abdomen areas as tested above.

\item Sing on 'ng' to put all air through nasopharynx

\item Eurythmy (actions as physical mnemonics for various notes, can't find these on internet?!)

\item Use the French time names, but h is better than s for rests, should be able to find them plus recordings on internet. (I like ta-ka-di-mi better!)

\item Metrical Hierarchy (stress beat 1, and also 3 when counting in fours)

\end{itemize}

Second Lesson

warm up (stretch vocal folds, hydrate them)

James thinks I'm a high baritone

comfy middle range top and bottom not so comfy

say \underline{'....ing ing ing ing....'} cause transition at \textbf{back of tongue, taps against soft palate}

this helps with high notes



low notes/''chest voice'', vocalis muscle in airstream is a relaxation oscillator (whoopee cushion!)



falsetto/extreme 'head voice'

crico aretenoid cartileges tighten, tenses lip of vocal fold (vocal ligament)

crico thyroid muscle

work together to tension vocal fold




\textbf {fragrant breath}

inhale nice smoke (clan, balkan sobranie)

\textbf {soft palate lifts}

\textbf {back of tongue lifts}

they are connected to the back of the thyroid cartilege

aiming for a non-breathy falsetto


I am covering the sound:

\textbf{externalise sound} / will it out there


use the light end of the voice,  thin fold, -> attain G4 easily



sing on a smile / lift facial muscles

relax jaw

wish to blend the two voices head and chest, so work on upper part









\end{document}
